\documentclass[12pt]{article}
\pagestyle{myheadings}
\def\ls{\vskip 7pt}                            %One line space.%
\textwidth=6.5in
\textheight=9.0in
\hoffset=0.0in
\voffset=0.0in
\setlength{\topmargin}{-0.5in}
\setlength{\textheight}{9.0in} 
\setlength{\textwidth}{6.5in} 
\setlength{\oddsidemargin}{-0.2in}
\setlength{\evensidemargin}{-0.2in}
\usepackage{amsmath}
%
\def\hfb          {\hfill\break}
\def\hsp          {\hskip 5pt}
\def\vs           {\vskip 10pt}
\def\mybase       {\baselineskip= 13pt plus 1pt minus 1pt}
\def\undertext#1  {$\underline{\smash{\hbox{#1}}}$}
\def\vp           {\hfill\vskip 12pt plus 1pt minus 1pt}

\parskip=10pt
\parindent=0pt

\pagestyle{empty}

\begin{document}


\centerline{\bf PHYS 20323/60323: Fall 2019- LaTeX Example}
\vskip 0.15in


\begin{enumerate}


\item {Consider a particle confined in a two-dimensional infinite square well}
\begin{align*}
    V(x,y)= \left\{ \frac{0}{\infty}\begin{array}{lr} 
      0\leq x \leq a   ,   0 < y < a \\
      0\leq x\leq 100 \\
   \end{array}
\end{align*}
\begin{comment}
\begin{displaymath}
   f(x) = \left\{
     \begin{array}{lr}
       1 & : x \in \mathbb{Q}\\
       0 & : x \notin \mathbb{Q}
     \end{array}
   \right.
\end{displaymath} 
\end{comment}
The eigenfunctions have the form:
\begin{align*}
    {\bf\Psi}(x,y)= \frac{2}{a}\sin{\left(\frac{n\pi x}{a}\right)} \sin{\left(\frac{m\pi y}{a}\right)}
\end{align*}
with the corresponding energies being given by:
\begin{align*}
    E_{nm}=\left(n^2 +m^2 \right) \frac{\pi^2 \hbar^2}{2ma^2}
\end{align*}
\begin{enumerate}
\item (5 points) What are the levels of degeneracy of the five lowest energy values?

\item (5 points) Consider a perturbation given by:
\begin{align*}
    \hat{H}'=a^2 V_0  \delta \left(x- \frac{a}{2}\right)\delta\left(y- \frac{a}{2}\right)
\end{align*}
Calculate the first order correction to the ground state energy.

\end{enumerate}

\vskip 0.15in

\item {\bf The following questions refer to stars in the Table Below.}\\
Note: There may be multiple answers.\\ 

\begin{tabular}{|c|c|c|c|c|c|}
\hline
 Name & Mass & Luminosity & Lifetime & Temperature & Radius \\ 
\hline
 Zeta & $60. {\it M_{sun}}$ & $10^6 {\it L_{sun}}$ &  $8.0*10^5$ years & & \\ 
 \hline
Epsilon & $6.0 {\it M_{sun}}$ & $10^3 {\it L_{sun}}$ & & $20,000$ K& \\
\hline
Delta & $2.0 {\it M_{sun}}$ &  &  $5.0*10^8$ years & & \\
\hline
Beta & $1.3 {\it M_{sun}}$ & $3.5 {\it L_{sun}}$ &  & & $2 {\it R_{sun}}$ \\
\hline
Alpha & $1.0 {\it M_{sun}}$ &  &   & & $1 {\it R_{sum}} $\\
\hline
Gamma & $0.7 {\it M_{sun}}$ &  &  $4.5*10^10$ years & $5000$ K& \\
\hline
\end{tabular}


\begin{enumerate}
\item(4 points) Which of these stars will produce a planetary nebula at the end of their life.\\

\vskip 0.15in

\item (4 points) Elements heavier than {\it Carbon} will be produced in which stars.
\end{enumerate}



%\end{enumerate}
\end{enumerate}



\end{document}

